
\chapter{2000多年前的貨幣戰爭,完爆!}
文/小馬鍋過河\footnote{https://kknews.cc/history/pj5ez.html}

近幾年來,經濟戰、貿易戰、金融戰和貨幣戰不絕於耳,對於剛解決溫飽問題的國人來說,新鮮刺激而又恐懼。
而且越來越多的精英投身金融,或為錢財,或為一展身手。殊不知,2700多年前的春秋時期,
中華大地上演了一部部精彩紛呈而又令人咋舌的貨幣戰爭。

\section{第一次貨幣戰爭:服帛降魯梁}

春秋時期,齊國、魯國、梁國都是山東地區的大國,彼此相鄰,戰爭不斷,各有勝負,
齊桓公無時不刻都在琢磨如何打垮這兩個國家?
有一次桓公對宰相管仲說:魯國、梁國天天在我眼皮子底下折騰,忍他們很久了,怎麼才能搞定他們?
管仲說:這個簡單,織綈業(紡織業的一種)是魯、梁兩國的支柱產業。您就帶頭穿綿綈的衣服,
下令左右近臣也穿,您可是齊國的天皇巨星啊,全國人民都追你,到時候老百姓也會跟著穿。
然後您再下令齊國人不許自己織綈,必須進口買綈就行了。

於是齊桓公就穿著綈做的衣服到處晃,還跑到泰山之南(魯國的家門口)炫,全國人民都爭相買綈效仿。
管仲讓魯、梁國的商人把綈出口到齊國,一千匹價格三百斤黃金,一萬匹三千斤。
魯、梁國靠出口創匯賺了大錢,國家都不用對老百姓收稅了,財政十分富裕。

十三個月後,管仲派特務去魯、梁國偵察,發現魯、梁國的人民太忙了,國家太繁榮了,
城市裡交通堵車,人都得慢慢挪著走。管仲說:哼,魯、梁國完了。桓公問:我靠,他們這麼繁榮,怎麼就完了?
管子說:請您以後不要再穿綈,也不要讓老百姓穿了,咱跟魯、梁國斷交,你看著吧。

十個月以後,管仲再次派特務去偵察,發現魯、梁國人餓死的很多,魯、梁國命令老百姓趕緊去把綈廠關掉改種糧食,
但是,糧食三兩個月根本長不成熟,魯國糧食價格漲到了齊國的十倍。
兩年後,魯國的老百姓60\%都移民到齊國了,三年以後,魯、梁國就投降了。


\section{第二次貨幣戰爭:制萊莒之謀}
收拾完魯國、梁國,齊桓公發現胳膊底下還有兩個小國,嘗到了經濟戰的甜頭,也懶得動兵了,
齊桓公問管仲:「萊、莒兩國對砍柴業(能源業)和農業都很重視,發展的都不錯,該怎樣對付他們?」
管仲說:「沒關係,萊、莒兩國的山上盛產柴薪(石油),您可以率領一批新兵蛋子在莊山煉銅鑄幣(央行加大馬力印錢),
高價收購萊國的柴薪。」萊國國君高興壞了,說;「金錢可是好東西,大家都喜歡。
柴薪是我國的特產,取之不盡,用柴薪出口創匯掙齊幣,齊國這傻帽兒,滅他指日可待。」
萊國隨即荒廢農業而專事打柴。管仲則撤回鑄錢士兵種地。兩年之後,桓公停止購柴。
萊國糧食價格暴漲,是齊國的三十七倍,70\%的兩國老百姓都移民到了齊國,萊國、莒兩國只有投降了。


\section{第三次貨幣戰爭:買鹿制楚}

齊桓公一直把南方的楚國看成王霸事業上的「假想敵」,整日裡都在琢磨如何削弱楚國。
但楚國的軍事戰鬥力很強,這讓齊桓公頭疼。他問管仲:「楚國是一個強國,其人民精通格鬥的技巧。
我們要舉兵討伐楚國,恐怕力不從心。這個楚國很麻煩,該怎麼辦呢?」
管仲說:「大王您出高價購買楚國特產的鹿吧,這一招准管用。」
管仲首先讓桓公通過民間買賣貯藏了國內糧食十分之六(儲存戰略物資),
其次派左司馬伯公率民夫到莊山鑄幣(央行開始印錢)。隨後桓公營建了百里鹿苑,
派人帶了二千萬錢去楚國大肆搜購活鹿,楚國活鹿的價格很快被抬高為八萬錢一頭。

楚王聽說後,樂了,對其宰相說:「那個金錢,是人都喜歡的(發現所有的國王怎麼都來這麼一句),
也是國家賴以生存的東西。而鹿,不過是禽獸而已,楚國多的是,即使都不要也無所謂。
現在齊國出那麼多錢來買我們不需要的東西,這是我們楚國的福氣啊!老天讓齊國這個傻冒來便宜我們,太好了!
趕快發布命令,讓老百姓趕緊捕捉活鹿,儘快把齊國手上的錢換過來!」

為炒做這一事件,管仲還煞有介事地對來自楚國的官方採購商人說:「你能給我弄來二十頭活鹿,
我就賞賜你黃金百斤;弄來二百頭,你就可以拿到千斤黃金了。楚國就算不向老百姓徵稅,財用也夠了。」
於是楚國上下都轟動了:無論官方還是民間,無論男女老少,全都來勁了,老百姓都放下手頭的農活,漫山遍野地去捕捉活鹿。

這個時候,管仲讓大臣隰朋悄悄地在齊、楚兩國的民間收購併囤積糧食:楚國靠賣活鹿賺的錢,
比往常多了五倍;齊國收購囤積的餘糧,也比往常多了五倍。

於是,管仲對齊桓公說:「好了,這下我們可以安心去攻打楚國了!」
齊桓公問:「為什麼?」管仲回答:「楚國拿了比往常多五倍的錢,卻誤了農時,糧食又不可能短短几個月時間成熟,
楚國一定會去收購糧食的。到時候我們封鎖邊境就行了。」齊桓公恍然,於是下令封閉與楚國的邊境。

結果楚國的米價瘋漲暴漲40倍,楚王派人四處買米,都被齊國截斷,最後逃往齊國的楚國難民多達本國人口的十分之四。
楚國元氣大傷,三年後向齊國屈服。

\section{第四次貨幣戰爭:買狐皮降代}
齊桓公和管仲搞經濟戰搞得不亦樂乎,桓公說:「要搞一個國家,首先看看他有什麼特殊資源,這次我要搞代國,
不過這個國家窮的鳥不拉屎,這個怎麼搞?」管仲回答說:「代國有一種狐白的毛皮,狐白適應寒暑變化,
六個月才出現一次(這是變異品種嗎),絕對是稀有資源,您可用高價去收購。」「這玩意也太稀缺了,代國會上當嗎?
這也不能發展成產業啊?」管仲又說;「您以天價收購,代國人忘其難得,喜其高價,一定會紛紛獵取,
您就派人帶錢去收購好了。」桓公說:「好吧。」

於是派中大夫王師北帶著錢到代國收購這狐白的皮張。代王聽到後欣喜若狂:「代國之所以比離枝國弱,
就是因為沒錢。現在齊國出錢收購我們狐白的皮張,我們撞大運啦。你趕緊命令百姓去搞這種毛皮,以換取齊國錢幣,
我要用這筆錢招降離枝國的百姓。」代國人果然因此而放下農業,走進山林,搜求狐白的皮張。

但時過兩年也沒有湊成一張,離枝國聽到以後,就侵入代國的北部。代王大為恐慌,又打不過,
只好率領土兵自願歸服齊國。齊國沒有花去一個錢,代國就降服了。

\section{第五次貨幣戰爭:衡山之謀}
衡山國盛產兵器,衡山利劍,天下無雙。齊桓公早就想征服衡山國,不過,要想以武力攻打衡山國,肯定要費一番功夫。
管仲又玩起了老套路,在起兵前一年就派人到衡山國高價收購兵器;十個月後,燕、代、秦等國都跟著到衡山國收購兵器,
可謂天下爭購。看到賺錢的情景,衡山國君得意道:天下各國都爭購我國兵器,
可使價錢提高二十倍以上,衡山國百姓於是紛紛放棄農業轉而打造兵器。

一年後,齊國派人到趙國購運糧食,趙國糧價每石十五錢,齊國卻按每石五十錢收購。
包括衡山國在內的諸國都運糧賣給齊國,就在各國為發財歡呼的時候,
齊國突然封閉關卡、停止收購糧食和衡山國兵器。

魯國趁火打劫,攻占衡山國南部,齊國攻打北部。此時,衡山國已經無糧可用,
兵器也差不多賣光了,又不能在別國買到糧食,在經濟和軍事兩個戰場上敗的精光,只得奉國降齊。


\section{以史為鏡,可以知興替}
齊國收服魯國、萊莒、楚、代、衡山,均是以輕重之策催垮對手的經濟。
其中心思想,就是利用「天下下我高,天下輕我重」的經濟原則,即以高價誘使敵方放棄本業,
追求某種產業的畸形利潤,使其變成單一經濟,產業結構畸形,然後突然改變國際貿易規則,
造成敵人經濟癱瘓,最終迫使其完全臣服。

在任何時代,一種商品價格暴漲都會帶來巨額利潤。這種利潤高的讓人炫目,除非這種利潤來自於國內壟斷性技術,
那是一定會出事情的。想想當年美國用「星球大戰計劃」拖垮蘇聯,用「廣場協議」設計日本,
先如今用石油狠削俄羅斯、沙特等石油出產國,美國用美元和石油玩轉全世界,如管仲用齊幣和糧食玩轉列國,
據說美國都有專門的機構研究\bookf{《管子》},想玩好貨幣戰爭,還需跟老祖宗學學!
