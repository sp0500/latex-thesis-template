\documentclass[8pt, %font size
a5paper, %paper type
twoside, % two sided printing
openright, % start new chapter on right side only ( inserts blank pages )
abstract=on, % Use an abstract
DIV=11,      % This parameter organizes the borders (detailed explanation at http://texdoc.net/texmf-dist/doc/latex/koma-script/scrguide.pdf
BCOR=8mm]{scrbook} % BCOR sets the space, used by the type of  book. (e.g. glued, hard cover..)
%scrreprt is used for larger texts with chapters (Master or Bachelor Thesis)
%scrartcl is used when there are no chapters ( for smaller paper)

\usepackage{fontspec}   %加這個就可以設定字體
\usepackage{xeCJK}      %讓中英文字體分開設置
%設定中文為系統上的字型,而英文不去更動,使用原TeX字型
\setCJKmainfont[AutoFakeBold=2,AutoFakeSlant=.4]{思源宋體}
\setCJKsansfont[AutoFakeBold=2,AutoFakeSlant=.4]{Noto Sans CJK TC}
\setCJKmonofont[AutoFakeBold=2,AutoFakeSlant=.4]{Noto Sans Mono CJK TC}
\XeTeXlinebreaklocale "zh"             %這兩行一定要加,中文才能自動換行
\XeTeXlinebreakskip = 0pt plus 1pt     %這兩行一定要加,中文才能自動換行
\leftskip=0pt                          %左右對齊
\rightskip=0pt plus 0cm                %左右對齊
\defaultCJKfontfeatures{AutoFakeBold=2,AutoFakeSlant=.4} %以後不用再設定粗斜
\newCJKfontfamily\Kai{I.Ngaan}         %定義指令\Kai則切換成顏楷體
\newCJKfontfamily\Hei{思源黑體}        %定義指令\Hei則切換成思源黑體
\newCJKfontfamily\NewMing{新細明體}    %定義指令\NewMing則切換成新細明體
\newCJKfontfamily\Newsong{I.BMing}
\newCJKfontfamily\Song{思源宋體}       %定義指令\Hei則切換成思源宋體
\newCJKfontfamily\SKai{全字庫正楷體}   %定義指令\SKai則切換成全字庫正楷體

\usepackage{graphicx}
\usepackage{float}
\usepackage{wrapfig}
\setlength\intextsep{5pt}

\newcommand{\bookf}[1]{{\Kai #1}}
\newcommand{\quotaf}[1]{{\cwHei #1}}


\usepackage[utf8]{inputenc}
\usepackage[english]{babel} % sets up english hyphenation
\usepackage{csquotes} % for language-dependent quotes in biblatex
\usepackage[unicode=true]{hyperref} % enables use of metadata for pdfs and hyperlinks within a document
\usepackage[natbib,maxnames=2,maxbibnames=100,style=authoryear-comp,uniquename=full,firstinits,doi=false,backend=biber,backref,hyperref]{biblatex} % advanced bibliography support
\usepackage[usenames,dvipsnames,hyperref]{xcolor} % enables more advanced color support for hyperref
\hypersetup{colorlinks=true, %flag for prints
    hidelinks,  % this option would hide links for the print version of your thesis
    linkcolor=red!35!black,    %definition of the link color
    citecolor=green!35!black,  %definition of the cite color
    urlcolor=magenta!35!black, %definition of the url color
    %pdfauthor=, % Optional: Specify the author of the pdf
    %pdftitle=   % Optional: Specify the title within the pdf
}      
\usepackage{subfiles} %This package is used for subfiles
\usepackage{tabu}     % provides advanced tables
\usepackage{array,multirow}
\bibliography{thesis.bib}% Include all the bibliography files
\usepackage{booktabs} % enables reference bookstyle tables
\usepackage[format=plain, labelfont=bf]{caption}
\usepackage{amsmath}
\usepackage[capitalize,noabbrev]{cleveref}
\usepackage{subcaption} % enables use multiple figures in a figure
\captionsetup{compatibility=false}
\usepackage{eurosym} %includes the euro symbol 
\usepackage{enumitem} % allows customization of enumeration and itemize environment
\usepackage{graphicx} % enables loading of graphics
\usepackage{tikz} % drawing vector graphics in latex
%\usepackage{parskip} %alternatively parskip replaces paragraph indentation by increased in-betweeen-paragraph linespacing 
\usepackage{setspace} % helps setup line spacing
%\onehalfspacing % increases linespacing to one and half
\usepackage{placeins} % provides FloatBarrier
%\usepackage[miktex]{gnuplottex} % gnuplot within latex. May be obsolete with pylab.
\usepackage[ruled,vlined]{algorithm2e} %algorithm package
\linespread{1.1} % Definition of the linespread
\usepackage[tbtags]{mathtools}
\DeclareMathOperator*{\somefunc}{somefunc}

%tikz helps to draw nice pictures with a lot of effort for advanced users
\usetikzlibrary{positioning,shapes,shadows,arrows, backgrounds}
\usepackage{verbatim}
\usepackage{tikz-3dplot}


%some definitions for the cref package
\crefname{algocf}{Algorithm}{Algorithms}
\crefname{table}{Table}{Tables}
\crefname{chapter}{Chapter}{Chapters}
\crefname{equation}{Equation}{Equations}
\crefname{section}{Section}{Sections}


\tikzset{
    tri/.style={
        draw,
        shape border rotate=90,
        isosceles triangle,
        isosceles triangle apex angle=60,
        node distance=1cm,
        minimum height=4em
    }
}



\usepackage{blindtext}
\begin{document}
    \frontmatter

    \begin{titlepage}
        \begin{center}

            % Upper part of the page. The '~' is needed because \\
            % only works if a paragraph has started.
            \includegraphics{img/logo2}\\[1cm]

            \textsc{\LARGE Rheinische\\[5mm] Friedrich-Wilhelms-Universität Bonn}\\[1.5cm]

            \textsc{\Large Master thesis}\\[1.5cm]

            % Title
            { \Large \bfseries Basic \LaTeX \, Template }\\[1.4cm]

            % Author and supervisor
            \begin{minipage}[t]{0.4\textwidth}
                \begin{flushleft} \large
                    \emph{Author:}\\
                    Max \textsc{Mustermann}
                \end{flushleft}
            \end{minipage}
            \begin{minipage}[t]{0.5\textwidth}
                \begin{flushright} \large
                    \emph{First Examiner:} \\
                    Prof.~Dr.~John \textsc{Doe} \\[0.5cm]
                    \emph{Second Examiner:} \\
                    Prof.~Dr.~John~\textsc{Doe} \\[0.5cm]

                    \emph{Advisor:} \\
                    John \textsc{Doe} \\[0.5cm]
                        %\emph{Abteilung:} \\
                        %Autonome Intelligente Systeme
                \end{flushright}
            \end{minipage}

            \vfill

            % Bottom of the page
            {\large Submitted:\hspace{1cm} \today}

        \end{center}
    \end{titlepage}

    \pagestyle{headings}  % switches on printing of running heads
    \title{Basic \LaTeX \, Template}
    %\subtitle{Master thesis}
    \author{Tobias Hartmann\\ \begin{minipage}{8cm}\centering \small Friedrich-Wilhelms-Universität Bonn\\ \small (group)\end{minipage}}

    \vspace{4cm}

    \cleardoublepage
    \thispagestyle{empty}
    {\noindent%
        \huge{\textbf{\textsf{Declaration of Authorship}}}
    }
    \vspace{2cm}
    \begin{flushleft}
        \noindent%
        I declare that the work presented here is original and the result of my own investigations.
        Formulations and ideas taken from other sources are cited as such.

        It has not been submitted, either in part or whole, for a degree at this or any other university.
    \end{flushleft}

    \vspace{8cm}
    \noindent%
    \rule[1em]{8em}{0.5pt}  \hfill \rule[1em]{8em}{0.5pt}\\ % This prints a line to write the date
    Location, Date \hfill Signature\\


    \cleardoublepage


    \chapter*{Abstract}
    \thispagestyle{empty}
    Describe your approach and results shortly in the abstract.
    The abstract should really already tell the reader what to expect.
    Do not try to  build suspension, this is not a Holywood  movie, it is a
    (your!) scientific thesis.  The abstract  is usually the last thing you
    write, even if it is the first thing you read here.

    \tableofcontents
    \listoffigures
    \listofalgorithms

    \newpage
    \mainmatter
    \subfile{content/introduction}

    \chapter{Language}

    \section{I or We?}

    Use the ``we'' form, even when  you're writing alone.  Imagine it to be
    the reader and  you, who are going through the  text together.  It gets
    you closer to your audience.

    \section{Quotes and Emphasis}
    Note that double "quotes" on  your keyboard are not typeset correctly. 
    English  quotes should  look like  ``this'', while  German quotes  are 
    typeset  like \glqq this\grqq.   Note the  difference.  Some  editors 
    insert the correct quotes automatically for you.                       

    Use  emphasis  sparingly.  It  clutters  the  text.  In  general,  use 
    \emph{italics}, which is the one standing out the least.  In captions, 
    it is sometimes useful to use  \textbf{bold} for the signal words left 
    and right, top and bottom: There, you want them to pop out.

    \section{Line Noise and Colloquial Expressions}

    Do  not use  fill words  which do  not carry  meaning.  Try  to be  as 
    specific as  possible.  This does  not mean  as short as  possible, it 
    means that you should have something to say when you write something.
    If you don't know what you want to say, think again, \emph{then} write.

    Don't use colloquial expressions.  In particular, don't use any of the 
    short forms  \emph{don't, aren't,  isn't, you're, we're,  \ldots}.  Do 
    not\footnote{this is much better!} use  ellipsis (\ldots) in the text, 
    as it looks like you're trailing off.                                  

    \chapter{Comments on Writing in LaTeX}

    \section{\LaTeX{}'s Wide Margins}

    Many students  complain  about  large  \LaTeX{}  standard  margins. 
    Typographers have a  rule, though: You shouldn't have  more than about 
    70 characters  per line (when  using single spacing).   Otherwise your 
    eyes have  trouble jumping to  the beginning  of the next  line.  This 
    limits  your column  width severely,  unless you  use a  big font.   A 
    thin  column width  results in  large  margins.  This  is (aside  from 
    portability) the reason  why books are smaller than A4  paper, and why 
    newspapers have multiple thin columns.                                 

    If  you want  to toy  with margins  nevertheless, consider  this: When 
    you  open  your  double-sided  printed thesis,  the  white  region  in 
    the  left/right  and  center  region should  have  equal  widths.   It 
    follows  that outside  margins  are larger  than  the margins  inside. 
    Consequentially, on  a left page,  left margins are larger  than right 
    (i.e.\ center) margins, and on a  right page, right margins are larger 
    than the left (i.e.\ center) ones. The  margin on the bottom should be 
    larger than  on the top.   These rules  are automatically used  by the 
    KomaScript classes  (scrartcl, scrbook, scrrprt).  Do  not meddle with 
    them.                                                                  
    
    To   meddle  with   the   margins,   use  the   DIV   option  of   the 
    \verb+documentclass+  command.    Larger  values  result   in  smaller 
    margins.  If you  want to bind your thesis, make  sure to include some 
    space for  it using BCOR\@.   Ask the print shop  how much to  use for 
    your  specific binding  needs.  Never,  ever, use  the \verb+geometry+ 
    package.                                                               

    \section{Paragraphs are Empty Lines}
    \label{sec:paras}

    Paragraphs are separated  by empty lines in the  \LaTeX{} code.  Never 
    use  the double  backslash for  this purpose.   \LaTeX{} inserts  page 
    breaks  preferably between  paragraphs.  A  double backslash  lets you 
    stay in the  same paragraph, so \LaTeX{} does not  know where it would 
    be good  to insert page  breaks, figures, tables, algorithms,  and the 
    like.                                                                  

    Many  people find  the  indentation  of the  first  line in  paragraph 
    odd.  Look into  a favorite book or newspaper of  yours and you'll see 
    that this way of introducing  new paragraphs is used everywhere.  Note 
    that it  allows you---even on the  top of a new  page---to see whether 
    the line is the first of a new paragraph or the continuation of an old 
    paragraph.                                                             

    \section{Typesetting Math as Part of Your Text}

    Formulas are part of the text, try  to use them like nouns and be sure 
    to include  punctuation to  keep the text  flowing.  For  example, the 
    pythagorean theorem,                                                   
    \begin{align}
        a^2 + b^2 = c^2,
    \end{align}
    is typically taught at school.

    Other random notes:

    \begin{itemize}
        \item
    Note that  as in \cref{sec:paras},  before and after a  formula, empty 
    lines will  add space  and indentation, and  possibly page  breaks, or 
    figures.  So  if a formula  is part of a  paragraph, do not  add empty 
    lines around it in the code.                                           

        \item
    Number all equations.  It makes it  easier to refer to them for others,
    even if you do not refer to them in your own text.

        \item
    While we're at  it, do not use the  \verb+equation+ or \verb+eqnarray+ 
    environments, use \verb+align+ instead.

        \item
    Take care to always put all  math in the math-environment, for example 
    the  variable $x$.   Note  that  it is  typeset  differently from  the 
    letter~x.

        \item
    Use  single-letter  names  for  variables   in  math  (math  is  not  a
    programming language!).   Function names  may have  multiple characters,
    such as $\tanh(x)$.  Note that they are not typeset in italics.  
    They  usually have  their own  commands, and  you can  define your  own
    functions, such as  $\somefunc(x)$.  If you don't do  this, the kerning
    will  be broken,  since \LaTeX{}  assumes that  you're multiplying  the
    variables  $s$, $o$,  $m$, $e$,  $f$, $u$,  $n$, and  $c$ and  typesets
    accordingly,  compare  $somefunc$  (math  mode,  without  spaces!)  and
    \emph{somefunc} (italics).

    \end{itemize}

    \section{Referencing Sections, Figures, etc.}
    Use  the \verb+cleverref+  package to  refer  to other  places in  the 
    document,  e.g.  \cref{chap:Introduction}.    The  package  allows  to 
    specify the reference style globally.  Therefore, it helps to refer in 
    a consistent  way.  Otherwise you  may quickly have different  ways of 
    referencing,   e.g.  \emph{Figure},   \emph{figure},  \emph{Fig.}   and
    \emph{fig.}.   The package  also ensures  that there  is no  linebreak 
    before the number of the figure/chapter etc.                           

    \begin{figure}
        \centering
        \begin{subfigure}[b]{0.3\textwidth}
            \includegraphics[width=\textwidth]{img/cat.jpeg}
            \caption{Nyan cat}
            \label{fig:cat1}
        \end{subfigure}%
        ~ %add desired spacing between images, e. g. ~, \quad, \qquad, \hfill etc.
        %(or a blank line to force the subfigure onto a new line)
        \begin{subfigure}[b]{0.3\textwidth}
            \includegraphics[width=\textwidth]{img/cat.jpeg}
            \caption{Nyan cat}
            \label{fig:cat2} 
        \end{subfigure}
        ~ %add desired spacing between images, e. g. ~, \quad, \qquad, \hfill etc.
        %(or a blank line to force the subfigure onto a new line)
        \begin{subfigure}[b]{0.3\textwidth}
            \includegraphics[width=\textwidth]{img/cat.jpeg}
            \caption{Nyan cat}
            \label{fig:cat3}  
        \end{subfigure}
        \caption[Nyan cats]{Pictures of Nyan cats}\label{fig:cats}
    \end{figure}



    \section{Citing Other Works}
    There are two ways to cite, which depends on whether the authors are an
    active part of your sentence:
    \begin{itemize}[noitemsep]
        \item \citet{muster} showed that\ldots{}
        \item This is currently a hot research topic \citep{muster}
    \end{itemize}
    In  \LaTeX{}, these  different  citation styles  are  reflected in  the
    \texttt{\textbackslash citet}  and  \texttt{\textbackslash citep}  commands. \textbf{Never  use  the  plain
    \texttt{\textbackslash cite}!} -- it is outdated.
    This distinction  is even more important  when you use a  numeric style
    for referencing, where in the second version only the number of the 
    reference is printed.   In the first case, a  number-only citation does
    not make much sense.

    In  your  bibliography, you  should  define  common strings,  such  as 
    conference  names, using  the  \verb|@string|  syntax.  Otherwise  you 
    might  quickly end  up with  a  lot of  different names  for the  same 
    conference, which is confusing and  inconsistent.  (Please have a look 
    at the bibliography file.)                                             

    \section{Floats}
    \subsection{Figures}
    You should  allow \LaTeX{} to place  the figures where it  wants.  The 
    same  goes  for  tables.   This  is  called  a  \verb+float+,  as  the 
    figure/table floats around.   Look in a well-typeset  book, and you'll 
    notice that figures aren't placed randomly in the text, they're rather 
    always  on the  top  of  the page  if  at  all possible. \LaTeX{}  has 
    options to  let figures float to  \emph{top} (\verb+t+), \emph{bottom} 
    (\verb+b+),   \emph{page  of   floats}   (\verb+p+)  and   \emph{here} 
    (\verb+h+).  The  default, \verb+tbp+, is  ok for most users,  in any 
    case, never use \verb+h+.  You can use the FloatBarrier command if you 
    want to make sure that a figure is within a specific section.          

    All figures have  a caption from which you can  understand most of the 
    figure content  and its significance.   They are also  referenced from 
    the main text (e.g., see \cref{fig:cats}).  Do not force linebreaks in 
    captions.  Do not deviate from these rules, they are strict.           

    Figures  should  provide a  short  name  for  every figure  in  square 
    brackets: $[$~$]$. If  you do not define a short  name, the whole text 
    of the caption will be displayed in the table of figures, which is 
    usually too long.
    
    Note that \cref{fig:cats} also shows how to use subfigures.



    \subsection{Tables}
    Never  use vertical  lines in  tables.  See  the documentation  of the 
    booktabs  package  for  an  explanation,  or  look  in  your  favorite 
    scientific  book/magazine to  understand  that this  is a  commonplace 
    rule.                                                                  

    A basic  example, taken  from the booktabs  documentation, is  given in
    \cref{tab::ex}.

    \begin{table}
        \centering
        \begin{tabular}{@{}lrr@{}} 
            \toprule
            \multicolumn{2}{c}{Education}\\ \cmidrule{1-2}
            Major & Duration & Income (\euro)\\ 
            \midrule 
            CompSci & 2 & 12,75 \\ \addlinespace
            MST & 6 & 8,20 \\ \addlinespace
            VWL & 14 & 10,00\\ 
            \bottomrule
        \end{tabular}
        \caption[Table Example]{A basic example from the booktabs package.}
        \label{tab::ex}
    \end{table}

    \paragraph{The tabu package}
    The tabu package provides some useful tools for advanced tables.


    \subsection{Algorithms}
    There are several packages to typeset algorithms.
    We recommend the \verb+algorithm2e+ package.
    In \cref{alg:exp} a simple example from the algorithm2e documentation is given.

    \begin{algorithm}[t]
        \SetAlgoLined
        \KwData{this text}
        \KwResult{how to write algorithm with \LaTeX2e }
        initialization\;
        \While{not at end of this document}{
            read current\;
            \eIf{understand}{
                go to next section\;
                current section becomes this one\;
            }{
                go back to the beginning of current section\;
            }
        }
        \caption{How to write algorithms (Small example from the algorithm2e documentation)}
        \label{alg:exp}
    \end{algorithm}

    \section{Miscellaneous}

    \subsection{Enumerations}

    Short enumerations look ugly in  standard \LaTeX{}, since they include 
    too much  space between the  items.  A ``normal'' enumerate  list with 
    very short paragraphs:                                                 
    \begin{enumerate}
        \item foo bar baz
        \item foo bar baz
        \item foo bar baz
    \end{enumerate}

    A list using enumitem to compress the inter item spaces, looks much better for short items:
    \begin{enumerate}[noitemsep]
        \item foo bar baz
        \item foo bar baz
        \item foo bar baz
    \end{enumerate}

    \subsection{Front, Main and Back Matters}

    Titlepage, lists of figures, table of contents page etc.\ are numbered 
    in  roman letters.   The introduction  should start  with page  $1$ in 
    arabic  numerals.   The document  class  \verb+scrbook+  does this  by 
    default,  when \textbackslash  frontmatter, \textbackslash  mainmatter 
    and \textbackslash backmatter are placed properly.                     



    \section{Compiling The Document}
    \LaTeX{}  is  a bit  archaic  to  work  with,  since with  every  pass 
    through  your document,  some intermediate  information is  written to 
    files,  which are  then processed  by  other programs.   Keep in  mind 
    that  everytime a  new  pass through  your  document incorporates  new 
    information (e.g. from BibTeX), page breaks may change, which requires
    an  additional pass  through the  document. Thus, you  usually have  to
    compile the document in three steps:
    \begin{enumerate}
        \item Compile the document using xelatex: xelatex thesis
        \item Run biber  (or if  you do  not have  it, bibtex)  to set  up
              bibliography: biber thesis
        \item Run xelatex  twice more to incorporate  bibtex references and
              to get the typesetting right.
    \end{enumerate}

    If  you know  what  you're doing  you  can get  away  with less.   Some
    programs parse the  output of \LaTeX{} to  determine whether additional
    passes are necessary.  Most GUI programs fall in this category, as do
    \verb+rubber+ and \verb+latexmk+.

    \subsection{Issues With This Template}
    \begin{description}
        \item [My references can't be found] 
            The line including the \texttt{biblatex} package says it should
            use \texttt{biber} as a backend.  This is a reimplementation of
            BibTeX. If you do not have \texttt{biber}, or your IDE does not
            support it, consider switching the backend to \texttt{bibtex}.
    \end{description}

\chapter{2000多年前的貨幣戰爭,完爆!}
文/小馬鍋過河\footnote{https://kknews.cc/history/pj5ez.html}

近幾年來,經濟戰、貿易戰、金融戰和貨幣戰不絕於耳,對於剛解決溫飽問題的國人來說,新鮮刺激而又恐懼。
而且越來越多的精英投身金融,或為錢財,或為一展身手。殊不知,2700多年前的春秋時期,
中華大地上演了一部部精彩紛呈而又令人咋舌的貨幣戰爭。

\section{第一次貨幣戰爭:服帛降魯梁}

春秋時期,齊國、魯國、梁國都是山東地區的大國,彼此相鄰,戰爭不斷,各有勝負,
齊桓公無時不刻都在琢磨如何打垮這兩個國家?
有一次桓公對宰相管仲說:魯國、梁國天天在我眼皮子底下折騰,忍他們很久了,怎麼才能搞定他們?
管仲說:這個簡單,織綈業(紡織業的一種)是魯、梁兩國的支柱產業。您就帶頭穿綿綈的衣服,
下令左右近臣也穿,您可是齊國的天皇巨星啊,全國人民都追你,到時候老百姓也會跟著穿。
然後您再下令齊國人不許自己織綈,必須進口買綈就行了。

於是齊桓公就穿著綈做的衣服到處晃,還跑到泰山之南(魯國的家門口)炫,全國人民都爭相買綈效仿。
管仲讓魯、梁國的商人把綈出口到齊國,一千匹價格三百斤黃金,一萬匹三千斤。
魯、梁國靠出口創匯賺了大錢,國家都不用對老百姓收稅了,財政十分富裕。

十三個月後,管仲派特務去魯、梁國偵察,發現魯、梁國的人民太忙了,國家太繁榮了,
城市裡交通堵車,人都得慢慢挪著走。管仲說:哼,魯、梁國完了。桓公問:我靠,他們這麼繁榮,怎麼就完了?
管子說:請您以後不要再穿綈,也不要讓老百姓穿了,咱跟魯、梁國斷交,你看著吧。

十個月以後,管仲再次派特務去偵察,發現魯、梁國人餓死的很多,魯、梁國命令老百姓趕緊去把綈廠關掉改種糧食,
但是,糧食三兩個月根本長不成熟,魯國糧食價格漲到了齊國的十倍。
兩年後,魯國的老百姓60\%都移民到齊國了,三年以後,魯、梁國就投降了。


\section{第二次貨幣戰爭:制萊莒之謀}
收拾完魯國、梁國,齊桓公發現胳膊底下還有兩個小國,嘗到了經濟戰的甜頭,也懶得動兵了,
齊桓公問管仲:「萊、莒兩國對砍柴業(能源業)和農業都很重視,發展的都不錯,該怎樣對付他們?」
管仲說:「沒關係,萊、莒兩國的山上盛產柴薪(石油),您可以率領一批新兵蛋子在莊山煉銅鑄幣(央行加大馬力印錢),
高價收購萊國的柴薪。」萊國國君高興壞了,說;「金錢可是好東西,大家都喜歡。
柴薪是我國的特產,取之不盡,用柴薪出口創匯掙齊幣,齊國這傻帽兒,滅他指日可待。」
萊國隨即荒廢農業而專事打柴。管仲則撤回鑄錢士兵種地。兩年之後,桓公停止購柴。
萊國糧食價格暴漲,是齊國的三十七倍,70\%的兩國老百姓都移民到了齊國,萊國、莒兩國只有投降了。


\section{第三次貨幣戰爭:買鹿制楚}

齊桓公一直把南方的楚國看成王霸事業上的「假想敵」,整日裡都在琢磨如何削弱楚國。
但楚國的軍事戰鬥力很強,這讓齊桓公頭疼。他問管仲:「楚國是一個強國,其人民精通格鬥的技巧。
我們要舉兵討伐楚國,恐怕力不從心。這個楚國很麻煩,該怎麼辦呢?」
管仲說:「大王您出高價購買楚國特產的鹿吧,這一招准管用。」
管仲首先讓桓公通過民間買賣貯藏了國內糧食十分之六(儲存戰略物資),
其次派左司馬伯公率民夫到莊山鑄幣(央行開始印錢)。隨後桓公營建了百里鹿苑,
派人帶了二千萬錢去楚國大肆搜購活鹿,楚國活鹿的價格很快被抬高為八萬錢一頭。

楚王聽說後,樂了,對其宰相說:「那個金錢,是人都喜歡的(發現所有的國王怎麼都來這麼一句),
也是國家賴以生存的東西。而鹿,不過是禽獸而已,楚國多的是,即使都不要也無所謂。
現在齊國出那麼多錢來買我們不需要的東西,這是我們楚國的福氣啊!老天讓齊國這個傻冒來便宜我們,太好了!
趕快發布命令,讓老百姓趕緊捕捉活鹿,儘快把齊國手上的錢換過來!」

為炒做這一事件,管仲還煞有介事地對來自楚國的官方採購商人說:「你能給我弄來二十頭活鹿,
我就賞賜你黃金百斤;弄來二百頭,你就可以拿到千斤黃金了。楚國就算不向老百姓徵稅,財用也夠了。」
於是楚國上下都轟動了:無論官方還是民間,無論男女老少,全都來勁了,老百姓都放下手頭的農活,漫山遍野地去捕捉活鹿。

這個時候,管仲讓大臣隰朋悄悄地在齊、楚兩國的民間收購併囤積糧食:楚國靠賣活鹿賺的錢,
比往常多了五倍;齊國收購囤積的餘糧,也比往常多了五倍。

於是,管仲對齊桓公說:「好了,這下我們可以安心去攻打楚國了!」
齊桓公問:「為什麼?」管仲回答:「楚國拿了比往常多五倍的錢,卻誤了農時,糧食又不可能短短几個月時間成熟,
楚國一定會去收購糧食的。到時候我們封鎖邊境就行了。」齊桓公恍然,於是下令封閉與楚國的邊境。

結果楚國的米價瘋漲暴漲40倍,楚王派人四處買米,都被齊國截斷,最後逃往齊國的楚國難民多達本國人口的十分之四。
楚國元氣大傷,三年後向齊國屈服。

\section{第四次貨幣戰爭:買狐皮降代}
齊桓公和管仲搞經濟戰搞得不亦樂乎,桓公說:「要搞一個國家,首先看看他有什麼特殊資源,這次我要搞代國,
不過這個國家窮的鳥不拉屎,這個怎麼搞?」管仲回答說:「代國有一種狐白的毛皮,狐白適應寒暑變化,
六個月才出現一次(這是變異品種嗎),絕對是稀有資源,您可用高價去收購。」「這玩意也太稀缺了,代國會上當嗎?
這也不能發展成產業啊?」管仲又說;「您以天價收購,代國人忘其難得,喜其高價,一定會紛紛獵取,
您就派人帶錢去收購好了。」桓公說:「好吧。」

於是派中大夫王師北帶著錢到代國收購這狐白的皮張。代王聽到後欣喜若狂:「代國之所以比離枝國弱,
就是因為沒錢。現在齊國出錢收購我們狐白的皮張,我們撞大運啦。你趕緊命令百姓去搞這種毛皮,以換取齊國錢幣,
我要用這筆錢招降離枝國的百姓。」代國人果然因此而放下農業,走進山林,搜求狐白的皮張。

但時過兩年也沒有湊成一張,離枝國聽到以後,就侵入代國的北部。代王大為恐慌,又打不過,
只好率領土兵自願歸服齊國。齊國沒有花去一個錢,代國就降服了。

\section{第五次貨幣戰爭:衡山之謀}
衡山國盛產兵器,衡山利劍,天下無雙。齊桓公早就想征服衡山國,不過,要想以武力攻打衡山國,肯定要費一番功夫。
管仲又玩起了老套路,在起兵前一年就派人到衡山國高價收購兵器;十個月後,燕、代、秦等國都跟著到衡山國收購兵器,
可謂天下爭購。看到賺錢的情景,衡山國君得意道:天下各國都爭購我國兵器,
可使價錢提高二十倍以上,衡山國百姓於是紛紛放棄農業轉而打造兵器。

一年後,齊國派人到趙國購運糧食,趙國糧價每石十五錢,齊國卻按每石五十錢收購。
包括衡山國在內的諸國都運糧賣給齊國,就在各國為發財歡呼的時候,
齊國突然封閉關卡、停止收購糧食和衡山國兵器。

魯國趁火打劫,攻占衡山國南部,齊國攻打北部。此時,衡山國已經無糧可用,
兵器也差不多賣光了,又不能在別國買到糧食,在經濟和軍事兩個戰場上敗的精光,只得奉國降齊。


\section{以史為鏡,可以知興替}
齊國收服魯國、萊莒、楚、代、衡山,均是以輕重之策催垮對手的經濟。
其中心思想,就是利用「天下下我高,天下輕我重」的經濟原則,即以高價誘使敵方放棄本業,
追求某種產業的畸形利潤,使其變成單一經濟,產業結構畸形,然後突然改變國際貿易規則,
造成敵人經濟癱瘓,最終迫使其完全臣服。

在任何時代,一種商品價格暴漲都會帶來巨額利潤。這種利潤高的讓人炫目,除非這種利潤來自於國內壟斷性技術,
那是一定會出事情的。想想當年美國用「星球大戰計劃」拖垮蘇聯,用「廣場協議」設計日本,
先如今用石油狠削俄羅斯、沙特等石油出產國,美國用美元和石油玩轉全世界,如管仲用齊幣和糧食玩轉列國,
據說美國都有專門的機構研究\bookf{《管子》},想玩好貨幣戰爭,還需跟老祖宗學學!


\blinddocument



    \backmatter

    \chapter{Appendix}
    Note that it the missing chapter number,  since it is behind
    the backmatter command.

    \FloatBarrier

    \begin{singlespacing}
        \printbibliography
    \end{singlespacing}

\end{document}

